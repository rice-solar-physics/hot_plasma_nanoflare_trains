\documentclass[preprint2,linenumbers]{aastex}

%Specify packages
\usepackage{amsmath}
\usepackage{tikz}
\usetikzlibrary{shapes.geometric, arrows}
\usepackage[all]{hypcap}
\usepackage[utf8]{inputenc} %force unicode in bibtex
\usepackage{enumitem}

%Define new commands as needed
\newcommand{\ang}{\AA~}
\defcitealias{barnes_inference_2016}{Paper I}
\renewcommand{\sectionautorefname}{Section}
\renewcommand{\subsectionautorefname}{Subsection}
%\linenumbers
%Set options for list spacing
\setenumerate{noitemsep}
%Set tikz options
\tikzstyle{box} = [rectangle, rounded corners, text centered, draw=black]
\tikzstyle{ghost} = [rectangle, rounded corners, text centered, draw=white]
\tikzstyle{arrow} = [thick, ->, >=stealth]

\begin{document}
	%Frontmatter
	\title{Inference of Heating Properties from ``Hot'' Non-flaring Plasmas in Active Region Cores II. Nanoflare Trains}
	\shorttitle{``Hot'' Non-flaring Plasmas II. Nanoflare Trains}
	\author{W. T. Barnes\altaffilmark{1}}
	\author{P. J. Cargill\altaffilmark{2,3}}
	\and
	\author{S. J. Bradshaw\altaffilmark{1}}
	%affiliations
	\altaffiltext{1}{Department of Physics \& Astronomy, Rice University, Houston, TX 77251-1892; will.t.barnes@rice.edu, stephen.bradshaw@rice.edu}
	\altaffiltext{2}{Space and Atmospheric Physics, The Blackett Laboratory, Imperial College, London SW7 2BW; p.cargill@imperial.ac.uk}
	\altaffiltext{3}{School of Mathematics and Statistics, University of St Andrews, St Andrews, Scotland KY16 9SS}
	%Abstract
	\begin{abstract}
		Faint, high-temperature emission in active region cores has long been predicted as a signature of nanoflare heating. However, the detection of such emission has proved difficult due to a combination of the efficiency of thermal conduction, non-equilibrium ionization, and inadequate instrument sensitivity. In this second paper in our series on ``hot'' non-flaring plasma in active regions, we investigate the influence of repeating nanoflares of varying frequency on the resulting emission measure distribution . We have used an efficient two-fluid hydrodynamic model to carry out a parameter exploration in preferentially heated species, heating event frequency, and the power-law index determining the distribution of event energies. We compute the emission measure distribution for each point in our multi-dimensional heating parameter space in an effort to understand how each of these variables impacts the observed emission. Additionally, we calculate several observables and compare their efficacy in capturing the character of both the hot and cool parts of the emission measure distribution. 
	\end{abstract}
	%Keywords
	\keywords{Sun:corona, Sun:nanoflares, plasmas, hydrodynamics}
	%Body
	\section{Introduction}
	\label{sec:intro}
	%
	\par\authorcomment1{Maybe intro is a bit lengthy...where to cut it down?}
	%
	\par The concept of heating the solar corona by nanoflares, first proposed by \citet{parker_nanoflares_1988}, has been developed extensively over the past two decades \citep[e.g.][]{cargill_implications_1994,cargill_nanoflare_2004,klimchuk_solving_2006}. The term \textit{nanoflare} has now become synonomous with impulsive heating in the energy range $10^{24}-10^{27}$ erg, with no specific assumption regarding the underlying physical mechanism (for example, small-scale magnetic reconnection or hydromagnetic wave dissipation).
	%
	In active region (AR) cores such as those we discuss in this paper, one strategy for constraining potential heating models is the analysis of the emission measure distribution as a function of temperature, $\mathrm{EM}(T)=\int n^2\mathrm{d}h$. \citet{cargill_implications_1994,cargill_nanoflare_2004} predicted that $\mathrm{EM}(T)$ resulting from nanoflare heating should be wide, with a maximum value at $T=T_m\sim10^{6.5}$ K and have a faint, high-temperature (8-10 MK) component. Below $T_m$, there is a scaling $\mathrm{EM}(T)\sim T^a$ over a temperature range $10^6\lesssim T\lesssim T_m$, a result first discussed by \citet{jordan_structure_1975}. Observations from the \textit{Hinode} spacecraft \citep{kosugi_hinode_2007} have shown that $2\lesssim a\lesssim5$, with $T_m\approx10^{6.5-6.6}$ \citep{warren_constraints_2011,warren_systematic_2012,winebarger_using_2011,tripathi_emission_2011,schmelz_cold_2012,del_zanna_evolution_2015}.
	%
	\par The emission component above $T_m$ has been the subject of less study, but is likely to be important as the so-called ``smoking gun'' of nanoflare heating since its properties may bear a close link to the actual heating. While many workers \citep{reale_evidence_2009,schmelz_hinode_2009,miceli_x-ray_2012,testa_hinode/eis_2012,del_zanna_elemental_2014,petralia_thermal_2014,schmelz_hot_2015} have claimed evidence of this hot, faint component of the emission measure, poor spectral resolution \citep{testa_temperature_2011,winebarger_defining_2012} and non-equilibrium ionization \citep{bradshaw_explosive_2006,reale_nonequilibrium_2008} have made a positive detection of nanoflare heating difficult. However, \citet{brosius_pervasive_2014} used observations from the \textit{EUNIS-13} sounding rocket to identify relatively faint emission from Fe XIX in a non-flaring active region (AR), suggesting temperatures of $\sim8.9$ MK.
	%
	\par A scaling has been claimed for hot emission with $T>T_m$ such that $\mathrm{EM}\propto T^{-b}$, with $b>0$. This fit is usually done in the range $T_m\lesssim T\lesssim10^{7.2}$. However, measured values of these ``hotward" slopes are poorly constrained due to both the low magnitude of emission and the lack of available spectroscopic data in this temperature range \citep{winebarger_defining_2012}. \citet{warren_systematic_2012}, find $7\lesssim b\lesssim10$, with uncertainties of $\pm2.5-3$, for 15 AR cores though \citet{del_zanna_elemental_2014}, using observations from the Solar Maximum Mission, claim larger values for $b$. It must be noted though that reconstructing $\mathrm{EM}(T)$ from spectroscropic and narrow-band observations is non-trivial, with differrent inversion methods often giving significantly different results \citep{landi_monte_2012,guennou_can_2013,aschwanden_benchmark_2015}.
	%
	\par An important parameter for any proposed coronal heating mechanism is the frequency of energy release along a single magnetic strand, where the observed loop comprises many such strands. Nanoflare heating can be classified as being either a \textit{high-} or \textit{low-frequency}. In the case of high-frequency (HF) heating, $t_N$, the time between successive events, is such that $t_N\ll\tau_{cool}$, where $\tau_{cool}$ is a characteristic loop cooling time, and in the case of low-frequency (LF) heating $t_N\gg\tau_{cool}$ \citep{mulu-moore_can_2011,warren_constraints_2011,bradshaw_diagnosing_2012,reep_diagnosing_2013,cargill_modelling_2015}. Steady heating is just high-frequency heating in the limit $t_N\to0$. While a determination of $t_N$ is of great importance, its measurement is challenging. For example, while direct observations of possible reconnection-associated heating through short timescale changes in loop structure and emission is feasible, as demonsrated by the Hi-C rocket flight \citep{cirtain_energy_2013,cargill_solar_2013}, longer duration observations are required to constrain $t_N$. The previously-mentioned difficulties in reconstructing $\mathrm{EM}(T)$ must also be bourne in mind. Efforts to measure the heating frequency using narrow-band observations of intensity fluctuations in AR cores  have proved similarly difficult \citep{ugarte-urra_determining_2014}.
	%
	\par The use of hydrodynamic loop models, combined with sophisticated forward modeling, is a useful method for assessing a wide variety of heating scenarios. Such models of nanoflare-heated loops have found emission measure slopes consistent with those derived from observations in the temperature range $T<T_m$. For example, while \citet{bradshaw_diagnosing_2012} found that the full range of $a$ could not be accounted for with low-frequency nanoflares, \citet{reep_diagnosing_2013} showed that using a tapered nanoflare train allowed for $0.9\lesssim a\lesssim4.5$. \citet{cargill_active_2014}, using a 0D loop model, investigated a large range of heating frequencies, $250<t_N<5000$ s, and found that only when $t_N$ was between a few hundred and 2000 seconds and proportional to the nanoflare energy could the full range of observed emission measure slopes be found. 
	%
	\par An analogous approach can be used to investigate the properties of the ``hot" coronal component expected from nanoflare heating, and is the subject of the present series of papers. In \citet{barnes_inference_2016} \citepalias[hereafter]{barnes_inference_2016}, we looked at the hot plasma properties due to a single isolated nanoflare. The effects of heating pulse duration, changes to conductive cooling due to heat flux limiting, differential heating of electrons and ions, and non-equilibrium ionization (NEI) were studied. It was shown that signatures of nanoflare heating are likely to be found in the temperature range $4\lesssim T\lesssim 10$ MK. The prospect of measurable signatures above $10$ MK was found to be diminished for short heating pulses (with duration $<100$ s), NEI, and differential heating of the ions rather than the electrons. It is important to stress for a single nanoflare that while the ``hot'' plasma is present, it cannot actually be detected.
	%
	\par However, nanoflare heating requires that we consider a ``train'' of nanoflares along a magnetic strand \citep{viall_patterns_2011,warren_constraints_2011,reep_diagnosing_2013,cargill_modelling_2015}. While a single nanoflare in the LF heating scenario will always occur in a plasma of characteristic AR density ($\sim10^9$ cm$^{-3}$), with significant consequences. In this paper, we use an efficient two-fluid hydrodynamic model to explore the effect of a nanoflare train with varying $t_N$ on $\mathrm{EM}(T)$, in particular for $T>T_m$. Preferential species heating, NEU, power-law nanoflare distributions, and the effects of a variable $t_N$ between events are considered and an emission measure ratio metric, similar to that discussed in \citet{brosius_pervasive_2014}, is used to characterize the various results. \autoref{sec:methods} discusses the numerical model we have used to conduct this study and the parameter space we have investigated. \autoref{sec:results} shows the resulting emission measure distributions and diagnostics for the entire parameter space. In \autoref{sec:discussion}, we discuss the impacts of two-fluid effects, pre-nanoflare density, and NEI on these calculated observables and how they may be interpreted in the context of nanoflare heating. Finally, \autoref{sec:conclusions} provides some concluding comments on our findings.
	%%
	\section{Methodology}
	\label{sec:methods}
	%
	\subsection{Numerical Model}
	\label{subsec:numerics}
	%
	\par 1D hydrodynamic models are excellent tools for computing field-aligned quantities in coronal loops. However, because of the small cell sizes needed to resolve the transition region and consequently small timesteps demanded by thermal conduction, the use of such models in large parameter space explorations is made impractical by long computational runtimes \citep{bradshaw_influence_2013}. We use the popular 0D enthalpy-based thermal evolution of loops (EBTEL) model \citep{klimchuk_highly_2008,cargill_enthalpy-based_2012,cargill_enthalpy-based_2012-1,cargill_modelling_2015} in order to efficiently simulate the evolution of a coronal loop over a large parameter space. This model, which has been successfully benchmarked against the 1D hydrodynamic HYDRAD code of \citet{bradshaw_influence_2013}, computes, with very low computational overhead, time-dependent, spatially-averaged loop quantities.
	%
	\par In order to treat the evolution of the electron and ion populations separately, we use a modified version of the usual EBTEL equations. This amounts to computing spatial averages of the two-fluid hydrodynamic equations over both the transition region and corona. A full description and derivation of these equations can be found in the appendix of \citetalias{barnes_inference_2016}. The relevant two-fluid pressure and density equations are,
	\begin{align}
		\frac{d}{dt}\bar{p}_e &= \frac{\gamma - 1}{L}\left\lbrack\psi_{TR} - (\mathcal{R}_{TR} + \mathcal{R}_C)\right\rbrack + \nonumber\\ &k_B\bar{n}\nu_{ei}(\bar{T}_i - \bar{T}_e) + (\gamma - 1)\bar{Q}_e,\label{eq:ebtel2fl_pe}\\
		\frac{d}{dt}\bar{p}_i &= -\frac{\gamma - 1}{L}\psi_{TR} + k_B\bar{n}\nu_{ei}(\bar{T}_e - \bar{T}_i) +\nonumber\\ &(\gamma - 1)\bar{Q}_i,\label{eq:ebtel2fl_pi}\\
		\frac{d}{dt}\bar{n} &= \frac{c_2(\gamma - 1)}{c_3\gamma Lk_B\bar{T}_e}\left(\psi_{TR} - F_{ce,0} - \mathcal{R}_{TR}\right),\label{eq:ebtel2fl_n}
	\end{align}
	%%
	where $c_2=\bar{T}_e/T_{e,a}\approx0.9$, $c_3=T_{e,0}/T_{e,a}\approx0.6$, $\nu_{ei}$ is the electron-ion binary Coulomb collision frequency and $\psi_{TR}$ is a term included to maintain charge and current and neutrality. Additionally, $c_1=\mathcal{R}_{TR}/\mathcal{R}_C$ and its formulation is discussed in \citet{cargill_enthalpy-based_2012} with additional corrections detailed in Appendix 1 of \citetalias{barnes_inference_2016}. These equations are closed by the equations of state $p_e=k_BnT_e$ and $p_i=k_BnT_i$. In the cases where we treat the plasma as a single-fluid, we use the original EBTEL model as described in \citet{klimchuk_highly_2008,cargill_enthalpy-based_2012}.
	%
	\par The loop is heated by a prescribed heating function, applied to either the electrons ($\bar{Q}_e$) or the ions ($\bar{Q}_i$). Both species cool through a combination of thermal conduction ($F_{ce,0},\,F_{ci,0}$) and an enthalpy flux to the lower atmosphere, with the electrons also undergoing radiative cooling ($\mathcal{R_C}$). In the case of conductive cooling, a flux limiter, $F=(1/2)fnk_BTV_e$, is imposed to mitigate runaway cooling in a low-density, high-temperature plasma. In all cases we use a saturation limit of $f=1$. See \citetalias{barnes_inference_2016} for a discussion of how $f$ is likely to effect the presence of hot emission in a nanoflare-heated plasma.
	%
	\subsection{Energy Budget}
	\label{subsec:params}
	%
	\begin{figure}
		\plotone{figures/heating_functions.pdf}
		\caption{\textbf{Top}: Uniform heating amplitudes for $t_N=1000$ s; \textbf{Middle:} Uniform heating amplitudes for $t_N=5000$ s; \textbf{Bottom:} Heating amplitudes drawn from a power-law distribution with index $\alpha=-1.5$. The events shown in red have wait times that depend on the previous event energy while the events shown in blue have uniform wait times. The mean wait time in both cases is $t_N=2000$ s.}
		\label{fig:heating_funcs}
	\end{figure}
	%
	\par We define our heating function in terms of a series of discrete heating events plus a static background heating to ensure that the loop does not drop to unphysically low temperatures and densities between events. For a triangular heating pulse of duration $\tau$ injected into a loop of half-length $L$ and cross-sectional area $A$, the total  event energy is $\varepsilon=LAH\tau/2$, where $H$ is the heating rate. Each model run will consist of $N$ heating events, each with peak amplitude $H_i$, and a steady background heating of $H_{bg}=3.5\times10^{-5}$ erg cm$^{-3}$ s$^{-1}$.
	%
	\par Recent observations have suggested that loops in AR cores are maintained at an equilibrium temperature of $T_{m}\approx4$ MK \citep{warren_constraints_2011,warren_systematic_2012}. Using our modified two-fluid EBTEL model, we have estimated the time-averaged volumetric heating rate needed to keep a loop of half-length $L=40$ Mm at $\bar{T}\approx4$ MK as  $H_{eq}\sim3.6\times10^{-3}$. In the single-fluid EBTEL model, this value is slightly lower because losses due to electron-ion collisions are ignored. Thus, to maintain an emission measure peaked about $T_{m}$, for triangular pulses, the individual event heating rates are constrained by 
	\begin{equation}
		\label{eq:heating_rate_constraint}
		H_{eq} = \frac{1}{t_{total}}\sum_{i=1}^N\int_{t_i}^{t_i+\tau}\mathrm{d}t~Q(t) = \frac{\tau}{2t_{total}}\sum_{i=1}^NH_i,
	\end{equation}
	where $t_{total}$ is the total simulation time. Note that if $H_i=H_0$ for all $i$, the heating rate for each event is $H_i=H_0=2t_{total}H_{eq}/N\tau$. Thus, for $L=40$ Mm, $A=10^{14}$ cm$^2$, the average energy per event for a loop heated by $N=20$ nanoflares in $t_{total}=8\times10^4$ s is $\varepsilon=LAt_{total}H_{eq}/N\approx5.8\times10^{24}$ erg, consistent with the energy budget of the Parker nanoflare model. 
	%
	\par We define the heating frequency in terms of the waiting time, $t_N$, between successive heating events. Following \citet{cargill_active_2014}, the range of waiting times is $250\le t_N\le5000$ s in increments of 250 s, for a total of 20 different possible heating frequencies. Additionally, $t_N$ can be written as $t_N=(t_{total}-N\tau)/N$, where we fix $t_{total}=8\times10^4$ s and $\tau=200$ s. Note that because $t_{total}$ and $\tau$ are fixed, as $t_N$ increases, $N$ decreases. Correspondingly, $\varepsilon_i=LA\tau H_i/2$, the energy injected per event, increases according to \autoref{eq:heating_rate_constraint} such that the total energy injected per run is constant.
	%
	\begin{figure}
		\centering
			\begin{tikzpicture}[node distance=2cm]
		%Draw the nodes
		\node (species) [ghost] {$\Sigma=\left\{
		\begin{array}{l}
			\mathrm{electron} \\
			\mathrm{ion} \\
			\mathrm{single}
		\end{array}
		\right.$};
		\node (ghost_ph) [ghost, below of=species] {};
		\node (alpha_pl) [ghost, left of=ghost_ph] {$\alpha=\left\{
		\begin{array}{l}
			-1.5 \\
			-2.0 \\
			-2.5
		\end{array}
		\right.$};
		\node (alpha_uni) [ghost, right of=ghost_ph] {uniform};
		\node (ghost_beta) [ghost, below of=alpha_pl]{};
		\node (beta_1) [ghost, right of=ghost_beta] {$\beta=1$};
		\node (beta_0) [ghost, left of=ghost_beta] {$\beta=0$};
		%Draw the arrows
		\draw [arrow] (species) -- (alpha_pl);
		\draw [arrow] (species) -- (alpha_uni);
		\draw [arrow] (alpha_pl) -- (beta_0);
		\draw [arrow] (alpha_pl) -- (beta_1);
		%\draw [arrow] (alpha_pl) -- (beta_2);	
	\end{tikzpicture}
		\caption{Total Parameter space covered. ``single'' indicates a single-fluid model. $\alpha$ is the power-law index and $\beta$ indicates the scaling in the relationship $Q\propto T_N^{\beta}$, where $\beta=0$ corresponds to the case where $t_N$ and the event energy are independent. Note that $(3~\alpha~\mathrm{values})\times(2~\beta~\mathrm{values})+\mathrm{uniform~heating}=$ 7 different types of heating functions.}
		\label{fig:parameter_space}
	\end{figure}
	%
	\par According to the nanoflare heating model of \citet{parker_nanoflares_1988}, turbulent loop footpoint motions twist and stress the field, leading to a buildup and subsequent release of energy. Following \citet{cargill_active_2014}, we let $\varepsilon_i\propto t_{N,i}^{\beta}$, where $\varepsilon_i,t_{N,i}$ are the total energy of event $i$ and waiting time following event $i$, respectively, and $\beta=1$ such that the event energy scales linearly with the waiting time. The reasoning for such an expression is as follows. Bursty, nanoflare heating is thought to arise from the stressing and subsequent relaxation of the coronal field. If a sufficient amount of energy is released into the loop, the field will need enough time to ``unravel'' and ``wind up'' again before the next event such that the subsequent waiting time is large. Conversely, if only a small amount of energy is released, the field will require a shorter unwinding time, resulting in a shorter interval between the subsequent events. Thus, this scaling provides a way to incorporate a more physically motivated heating function into a hydrodynamic model which cannot self-consistently determine the heat input based on the evolving magnetic field. \autoref{fig:heating_funcs} shows the various heating functions used for several example $t_N$ values.
	%
	\subsection{Heating Statistics}
	\label{subsec:heating_stats}
	%
	\par We compute the peak heating rate per event in two different ways: 1) the heating rate is uniform such that $H_i=H_0$ for all $i$ and 2) $H_i$ is chosen from a power-law distribution with index $\alpha$ where $\alpha=-1.5,-2.0,$ or $-2.5$. For the second case, it should be noted that, when $t_N\approx5000$ s, $N\sim16$ events, meaning the events from a single run do not accurately represent the distribution of index $\alpha$. Thus, a sufficiently large number of runs, $N_{R}$, are computed for each $t_N$ to ensure that the total number of events is $N_{tot}=N\times N_{R}\sim10^4$ such that the distribution is well-represented. \autoref{fig:parameter_space} shows the parameter space we will explore. For each set of parameters and waiting time $t_N$, we compute the resulting emission measure distribution for $N$ events in a period $t_{total}$. This procedure is repeated $N_R$ times until $N\times N_R\sim10^4$ is satisfied. Thus, when $t_N=5000$ s and $N\sim16$, $N_R=625$, meaning the model is run 625 times with a heating frequency of $t_N=5000$ s in order to properly fill out the event energy distribution.
	%
	\subsection{Non-equilibrium Ionization}
	\label{subsec:nei}
	%%
	\par When considering the role of nanoflares in the production of hot plasma in AR cores, it is important to take non-equilibrium ionization (NEI) into account \citep{bradshaw_explosive_2006,reale_nonequilibrium_2008}. In a steady heating scenario, the ionization state is an adequate measure of the electron plasma temperature. Because the heating timescale is long (effectively infinite), the ionization state has plenty of time to come into equilibrium with the electron temperature. 
	%
	\par In a nanoflare train, when the heating frequency is high, the loop is not allowed to drain or cool sufficiently between events, meaning the ionization state is kept at or near equilibrium. However, as the heating frequency decreases, the loop is allowed to cool and drain more and more during the inter-event period. If the heating occurs on a short enough timescale, the ionization state will not be able to reach equilibrium with the electron plasma before the loop undergoes rapid cooling by thermal conduction. Furthermore, if the frequency is sufficiently low so as to allow the loop to drain during the inter-event period, the ionization equilibrium timescale will increase. Thus, in the context of intermediate- to low-frequency nanoflares, NEI should be considered.
	%
	\par As in \citetalias{barnes_inference_2016}, we use the numerical code\footnote{This code has been made freely available by the author and can be downloaded at: \url{https://github.com/rice-solar-physics/IonPopSolver}.} outlined in \citet{bradshaw_numerical_2009} to asses the impact of NEI on our results. Given a temperature ($T(t)$) and density ($n(t)$) profile from EBTEL, we compute the non-equilibrium ionization states for Fe IX through XXVII and the corresponding effective electron temperature, $T_{eff}$, that would be inferred by assuming ionization equilibrium. Using $T_{eff}$, we are then able to compute a corresponding NEI emission measure distribution, $\mathrm{EM}(T_{eff})$.
	%%
	\section{Results}
	\label{sec:results}
	%
	\par We now show the results of our nanoflare train simulations for each point in our multidimensional parameter space: species heated (single-fluid, electron or ion), power-law index ($\alpha$), heating frequency ($t_N$), and waiting-time/event energy relationship ($\beta$). In each 0D hydrodynamic simulation, a loop of half-length $L=40$ Mm is heated by $N$ triangular events of duration $\tau=200$ s and peak heating rate $H_i$ for a duration of $t_{total}=8\times10^4$ s. The average interval between subsequent events is $t_N$ (in the uniform and $\beta=0$ cases, $t_{N,i}=t_N$ exactly for all $i$). We focus primarily on the emission measure distribution, $\mathrm{EM}(T)$, and observables typically calculated from $\mathrm{EM}(T)$. In all cases, the coronal emission measure is calculated according to the method outlined in section 3 of \citetalias{barnes_inference_2016}. The corresponding NEI results, $\mathrm{EM}(T_{eff})$, are calculated similarly, but using $T_{eff}$ (see \autoref{subsec:nei}) instead of $T$. All results were processed using the IPython system for interactive scientific computing in Python \citep{perez_ipython:_2007} as well as the NumPy and Scipy numerical and scientific Python libraries \citep{van_der_walt_numpy_2011}. All results were visualized using the matplotlib graphics library \citep{hunter_matplotlib:_2007}.
	%%
	\subsection{Emission Measure Distributions}
	\label{subsec:em_dist}
	%
	\begin{figure*}[t]
		\plotone{figures/em_grid_single_a25.pdf}
		\caption{Emission measure distributions for waiting-times $t_N=250,750,1500,2500,3750,5000$ s in the single-fluid case. The three types of heating functions shown are uniform heating rates (red), heating rates chosen from a power-law distribution of $\alpha=-2.5$ (blue), and heating rates chosen from a power-law distribution of $\alpha=-2.5$ where the time between successive events is proportional to the heating rate of the preceding event (green). The solid lines in the two power-law cases show the mean $\mathrm{EM}(T)$ over $N_R$ runs and the shading indicates $1\sigma$ from the mean. The dashed lines denote the corresponding $\mathrm{EM}(T_{eff})$ distribution. The standard deviation is not included in the NEI results.}
		\label{fig:single_em}
	\end{figure*}
	%
	\begin{figure*}[t]
		\plotone{figures/em_grid_electron_a25.pdf}
		\caption{Same as \autoref{fig:single_em}, but for the case where only the electrons are heated.}
		\label{fig:el_em}
	\end{figure*}
	%
	\begin{figure*}
		\plotone{figures/em_grid_ion_a25.pdf}
		\caption{Same as \autoref{fig:single_em}, but for the case where only the ions are heated.}
		\label{fig:ion_em}
	\end{figure*}
	%
	\par In our first set of results, we compare $\mathrm{EM}(T)$ for three different types of heating functions, across six different heating frequencies. \autoref{fig:single_em}, \autoref{fig:el_em}, and \autoref{fig:ion_em} show the emission measure distributions in the single-fluid case, electron heating case, and ion heating case, respectively. Each panel in each figure corresponds to a different waiting time ($t_N$) and includes three different types of heating functions: uniform heating events (red), events chosen from a power-law distribution of index $\alpha=-2.5$ ($\beta=0$ case, blue), and events chosen from a power-law distribution of index $\alpha=-2.5$ where the time between successive events depends on the heating rate of the preceding event ($\beta=1$ case, green). Note that in all three figures, we show the results for only a single power-law index, $\alpha=2.5$. Furthermore, the dashed lines denote the corresponding NEI cases, $\mathrm{EM}(T_{eff})$. The cases shown in \citetalias{barnes_inference_2016} correspond approximately to the red curves in the lower right panels ($t_N=5000$ s) of each of the three figures since the loop is allowed to cool and drain completely before reheating and a single nanoflare energy is used. 
	%
	\par In \autoref{subsec:heating_stats}, we noted that for heating functions using a power-law energy distribution, for each $t_N$, we run the model $N_R$ times. Thus, for each point in our parameter space, we produce $N_R$ $\mathrm{EM}(T)$ curves. In order to present our results compactly, the solid lines in \autoref{fig:single_em}, \autoref{fig:el_em}, and \autoref{fig:ion_em} each show the mean $\mathrm{EM}(T)$ over all $N_R$ curves. The shading represents $1\sigma$ from the mean. In this way, we account for the variations that may occur because of a lack/excess of strong heating events due to limited sampling from the distribution.
	%
	\par We look first at $\mathrm{EM}(T)$ for the single-fluid case, \autoref{fig:single_em}. Firstly, as expected from \citet{cargill_active_2014}, as $t_N$ increases, $\mathrm{EM}(T)$ widens, extending to both cooler ($<4$ MK) and hotter ($>4$ MK) temperatures. The extension toward cooler temperatures arises because as $t_N$ increases there is more time between successive heating events so that the loop so that the loops cools to lower temperatures before being reheated. The dependence on $\alpha$ and $\beta$ is similar to that described in \citet{cargill_active_2014}. The uniform (red) and $\beta=0$ (blue) $\mathrm{EM}(T)$ curves evolve similarly as they extend to cooler temperatures with increasing $t_N$ while the $\beta=1$ curves (green) extend to cooler temperatures much more rapidly. For example, at $t_N=1500$ s, both the uniform and $\beta=0$ cases show little to no emission below 2 MK while the $\beta=1$ cases extends to temperatures well below 1 MK \citep{cargill_active_2014}. As in \citetalias{barnes_inference_2016}, the results below $T_m$ are relatively independent of which species is heated.
	%
	\par The behavior of $\mathrm{EM}(T)$ above $T_m$ is more complicated. For small $t_N$, the emission measure distribution falls off sharply on the hot side for a uniform nanoflare train but a power law distribution leads to a broader distribution above $T_m$. This just reflects the different initial temperatures generated with a power-law distribution since $T \simeq H^{2/7}$. As $t_N$ increases, the distribution for uniform heating gradually broadens as the initial temperature rises due to the lower density in which the heating occurs. A similar broadening occurs for the power laws with the $\beta=0$ and $\beta=1$ results showing little difference. Note that the $\beta=0$ curve is barely visible as it overlaps almost completely with the $\beta=1$ curve. Especially interesting in this case are the results with NEI included. For a uniform nanoflare train, NEI plays no role up to $t_N$ = 2500 sec, but above that it restricts the temperatures that can be detected, as shown in \citetalias{barnes_inference_2016}. This hot emission is relocated to cooler temperatures, resulting in a ``bump'' in the emission measure distribution near 10 MK. On the other hand, NEI plays almost no role in the power law distributions, in both the $\beta=1$ and $\beta=0$ cases.
	%
	\par From these initial results, there is an important difference in the information available in potential observations. Below $T_m$ the value of $a$ requires a specific form of the waiting time between nanoflares which in turn informs about the nature of energy release \citep{cargill_active_2014}. Above $T_m$, there is no information about the role of a waiting time, but the emission measure distribution does inform about the need for a power-law energy distribution and the density in which the nanoflare occurs. Taken together, the overall conclusion is the same: nanoflares require an \textit{intermediate frequency}. 
	%
	\par For electron heating, the $\mathrm{EM}(T)$ curves for the different types of heating functions shown in \autoref{fig:el_em} evolve similarly to those shown in \autoref{fig:single_em}, especially on the cool side where the density is sufficiently high such that the electrons and the ions to equilibrate. On the hot side, for $t_N\le750$ s, the electron and single-fluid cases are quite similar. However, for $t_N\ge1500$ s, in the electron heating case, $\mathrm{EM}(T)$ steepens just above 4 MK and then flattens out near 10 MK. This change in shape is most obvious in the uniform heating case where a distinct ``hot shoulder'' forms just above 10 MK. In the power-law cases, this feature is less pronounced though $\mathrm{EM}(T)$ extends to slightly higher temperatures. The conclusions regarding the role of NEI are similar to those in the single-fluid case: it is not a factor when the heating rates are determined by a power-law. In the uniform case, the hot emission is again truncated and relocated to cooler temperatures, with more pronounced ``bump'' in the $\mathrm{EM}(T)$ near $10$ MK.
	%
	\par When only the ions are heated (\autoref{fig:ion_em}), the cool side of the $\mathrm{EM}(T)$ is again very similar to both the single-fluid and electron heating cases due to equilibration. On the hot side, for intermediate to low heating frequencies (i.e. $t_N\ge1500$ s), the $\mathrm{EM}(T)$ in the uniform heating case is truncated below 10 MK and in the power-law cases extends to just above 10 MK for the lowest heating frequency ($t_N=5000$ s). This cutoff at lower temperatures is due to the fact that electrons cannot ``see'' the ions until they have cooled well below their peak temperature. This is discussed in \citetalias{barnes_inference_2016} though in the single-pulse cases, the cutoff occured at lower temperatures. Additionally, in both the uniform and power-law cases, the peak of $\mathrm{EM}(T)$ is wider for these low frequencies compared to those cases shown in the lower right panels of \autoref{fig:single_em} and \autoref{fig:el_em}. Regarding the role of NEI, there is no significant difference between $\mathrm{EM}(T)$ and $\mathrm{EM}(T_{eff})$ for either the uniform or power-law cases. Here the electrons are essentially heated on a timescale dictated by the Coulomb collision frequency which is slow enough to ensure ionization equilibrium throughout the heating and conductive cooling phases.
	%
	\subsection{Pre-nanoflare Density}
	\label{subsec:pre_nanoflare_density}
	%
	\begin{figure*}
		\plotone{figures/nT_sample_curves_tn2500_electron.pdf}
		\caption{Example heating (top), temperature (middle), and density (bottom) profiles for the case in which only the electrons are heated with an intermediate heating frequency of $t_N=2500$ s. The three curves shown in each panel correspond to uniform heating rates (red), heating rates chosen from a power-law distribution of $\alpha=-2.5$ (blue), and heating rates chosen from a power-law distribution of $\alpha=-2.5$ where the time between successive events is proportional to the heating rate of the preceding event (green).}
		\label{fig:nT_sample_profiles}
	\end{figure*}
	%
	\par In \citet{cargill_active_2014}, \citetalias{barnes_inference_2016}, and \autoref{subsec:em_dist}, we have suggested that the plasma density prior to the nanoflare occuring is a crucial parameter in determining the emission measure distribution. This arises in two distinct ways. Below $T_m$, the temperature and density at which the nanoflare occurs cuts off the emission at lower temperatures. When combined with an energy-dependent waiting time, this can lead to a range of EM slopes in this region \citep{cargill_active_2014}. Above $T_m$, the initial density determines the temperature increase due to the nanoflare, how quickly the initial hot plasma cools, and whether NEI effects are important. We now examine this further.
	% 
	\par In the single-fluid and electron heating cases, while $\mathrm{EM}(T)$ in the uniform and power-law heating cases generally agree for low-frequency heating ($t_N=5000$ s), for intermediate frequencies ($t_N\approx750-2500$ s), the power-law cases show an enhanced high-temperature component compared to the uniform case as seen in \autoref{fig:single_em} and \autoref{fig:el_em}. \autoref{fig:nT_sample_profiles} shows sample heating, temperature, and density profiles for an intermediate heating frequency ($t_N=2500$ s), in the case where only the electrons are heated, for the three different types of heating functions. In the uniform heating rate case (red), each event has a maximum heating rate of $H_0$ such that the loop undergoes $N\approx30$ identical heating and cooling cycles, each time reaching a maximum temperature and density of $T_{max,0}$ and $n_{max,0}$, respectively. 
	%
	\par In comparing various heating models, we insist that the total energy injected into the loop is the same for each run (see \autoref{eq:heating_rate_constraint}). When the nanoflare heating rates are distributed according to a power-law, there will be many events where $H_i<H_0$ and a few events where $H_i\gg H_0$. These few high energy events lead to $T\gg T_{max,0}$ (blue and green curves) as seen in the middle panel of \autoref{fig:nT_sample_profiles}. Because these events are injected into a plasma that is sufficiently dense due to the draining and cooling times being longer than the time since the previous event, the emission measure is able to ``see'' these hot temperatures, resulting in a $>10$ MK component of $\mathrm{EM}(T)$ (see lower left panel of \autoref{fig:single_em} and \autoref{fig:el_em}). In the uniform case, $T_{max,0}<10$ MK such that $\mathrm{EM}(T)$ has a steep cutoff right at 10 MK. Additionally, we note that the only effect of the longer wait times in the $\beta=1$ case as compared to the $\beta=0$ case is an extended cool emission measure at intermediation frequencies (e.g. $t_N=2500$ s); the hot part of $\mathrm{EM}(T)$ for both power-law cases is nearly identical. Thus, the hot and cool sides of the emission measure distribution are largely isolated from one another as they are determined by separate phases of the loop evolution.
	%
	\subsection{Hot Plasma Diagnostics}
	\label{subsec:diagnostics}
	%%
	\subsubsection{Emission Measure Slope}
	\label{subsubsec:em_slope}
	%%
	\begin{figure*}
		\plotone{figures/em_slope_varying_bounds.pdf}
		\caption{Fits to a sample emission measure distribution constructed from a loop plasma in which only the electrons were heated by events chosen from a power-law distribution with $\alpha=-2.5$ and equally spaced by an interval of $t_N=5000$ s. \textbf{Left:} $\mathrm{EM}(T)$ slope as a function of upper bound on fit interval for both the hot (red) and cool (blue) side of $\mathrm{EM}(T)$. The shading denotes the uncertainty of the fit. The bottom axis corresponds to the varying upper limit on the fit to the cool side while the top axis corresponds to the varying upper limit on the fit to the hot side. \textbf{Right:} $\mathrm{EM}(T)$ with the overlaid hot (red) and cool (blue) fit lines whose slopes correspond to those shown on the left. The cool power-law fits describe $\mathrm{EM}(T)$ for $T<4$ MK quite well while a similar fit on the hot side fails to accurately describe the shape of the emission measure distribution for $T>4$ MK.}
		\label{fig:em_slope_varying_bounds}
	\end{figure*}
	%%
	%
	\par We now examine several observables often used to characterize the emission measure distribution. The most common observable is the emission measure slope $a$ such that $\mathrm{EM}\propto T^a$ for $10^{5.5}\le T\le10^{6.6}$ K. Both observational and modeling studies have found that $2\lesssim a\lesssim5$ \citep[see Table 3 of][]{bradshaw_diagnosing_2012} and in particular, \citet{cargill_active_2014} found that a heating function of the form $t_N\propto\varepsilon$ was needed in order to account for this range of slopes. Additionally, a similar scaling of $\mathrm{EM}\propto T^{-b}$ for $10^{6.6}\le T\le10^{7.0}$ has been claimed though measurements of $b$ have been subject to large uncertainties \citep{warren_systematic_2012}.
	%
	\par\autoref{fig:em_slope_varying_bounds} shows an example of how both $a$ and $b$ can be calculated from the cool and hot sides of $\mathrm{EM}(T)$, respectively. We select a single sample run from our parameter space in which only the electrons are heated by nanoflares from a power-law distribution of $\alpha=-2.5$ and spaced uniformly by an interval of $t_N=5000$ s. We calculate the resulting $\mathrm{EM}(T)$ and fit $\log{\mathrm{EM}}$ to $a\log{T}$ on $\log{T_{c,min}}<\log{T}<\log{T_{c,max}}$ and $-b\log{T}$ on $\log{T_{h,min}}<\log{T}<\log{T_{h,max}}$ using the Levenburg-Marquardt algorithm for least-squares curve fitting as implemented in the SciPy scientific Python package \citep{van_der_walt_numpy_2011}. We fix the lower limit on each interval such that $T_{c,min}=10^{5.7}$ K and $T_{h,min} = 10^{6.7}$  K and vary the upper limits over $10^{6.1}<T_{c,max}<10^{6.5}$ K and $10^{6.8}<T_{h,max}<10^{7.2}$ K. The left panel of \autoref{fig:em_slope_varying_bounds} shows $a$ (blue) and $b$ (red) as a function upper limit of the fit interval, $T_{c,max}$ (bottom axis) and $T_{h,max}$ (top axis), respectively. The shading denotes the uncertainty of the fit. The right panel of \autoref{fig:em_slope_varying_bounds} shows the resulting fit lines superimposed on the emission measure distribution.
	%
	\par From the left panel of \autoref{fig:em_slope_varying_bounds}, we see that, while $a$ is relatively insensitive to the fit interval, $b$ varies between approximately 2 and 4.5 depending on the choice of bounds. Furthermore, the uncertainty in the fitting procedure for $b$ is relatively large, with the average uncertainty over the entire range of $T_{h,max}$ being $\bar{\sigma}_b\approx0.17$. Contrastingly, we find that $a\approx2.3$ with little variation over all values of $T_{c,max}$ considered here and that $\bar{\sigma}_a\approx0.018$, nearly an order of magnitude smaller than $\bar{\sigma}_b$. The overlaid fit lines in the right panel of \autoref{fig:em_slope_varying_bounds} similarly show that while $\log{\mathrm{EM}}$ is roughly linear over $5.7<\log{T}<6.5$, this is not the case for the interval $6.7<\log{T}<7.2$. In particular, a function of the form $T^{-b}$ cannot describe the hot shoulder in the emission measure distribution near $10^{7.1}$ K.
	%
	\par\authorcomment1{More discussion here? Some conclusions? What else do we need to say? Lead in to the next point a bit better? Maybe some discussion of why the slopes are different? What slopes we should expect? Refer to Paper 1...}
	%
	\subsubsection{Emission Measure Ratio}
	\label{subsubsec:em_ratio}
	%%
	\begin{figure*}
		\plottwo{figures/ratio_hist_alpha_electron_T2.pdf}{figures/ratio_hist_twait_electron_T2.pdf}
		\caption{Histograms of emission measure ratios for all heating function types and heating frequencies for the case in which only the electrons are heated. In both panels each histogram is normalized such that for each distribution $P(x)$, $\int_{-\infty}^{\infty}\mathrm{d}x~P(x)=1$ and the bin widths are calculated using the well-known Freedman-Diaconis formula \citep{freedman_histogram_1981}. \textbf{Left:} Emission measure ratios separated by heating function type. \textbf{Right:} Emission measure ratios separated by waiting time, $t_N$. For aesthetic purposes, only five values of $t_N$ are shown.}
		\label{fig:em_ratio_hist_el}
	\end{figure*}
	%%
	\begin{figure*}
		\plottwo{figures/ratio_hist_alpha_ion_T2.pdf}{figures/ratio_hist_twait_ion_T2.pdf}
		\caption{Same as \autoref{fig:em_ratio_hist_el}, but for the case where only the ions are heated.}
		\label{fig:em_ratio_hist_ion}	
	\end{figure*}
	%%
	\par\citet{brosius_pervasive_2014}, using observations of an active region from the \textit{EUNIS-13} sounding rocket, found that the intensity ratio of Fe XIX (formed at $T\approx10^{6.95}$ K) to Fe XII (formed at $T\approx10^{6.2}$ K) is $\sim0.59$ inside the AR core as compared to $\sim0.076$ outside, providing possible evidence for impulsive heating. As a proxy for this intensity ratio and an alternative to the hot emission measure slope $b$, we propose using an emission measure ratio, $\mathrm{EM}(T_{hot})/\mathrm{EM}(T_{cool})$. Unlike an emission measure slope, a simple emission measure ratio gives information about the emission at only a single ``hot'' temperature relative to a single ``cool'' temperature. We choose $T_{hot}=10^{6.95}$ K and $T_{cool}=10^{6.3}\approx2\times10^6$ K, the latter slightly higher (closer to $T_{m}\approx4$ MK) than the formation temperature of Fe XII such that the emission measure ratio is less sensitive to variations in the cool emission and thus a more accurate measure of changes in the amount of hot plasma. The ratio of other emission lines could also be used. 
	%
	\par Such an emission measure ratio also provides a way to compare in a concise way every point in our multidimensional parameter space while acknowledging that we are reducing $8\times10^4$ s of loop evolution to a single value. The left panel of \autoref{fig:em_ratio_hist_el} shows the case in which only the electrons are heated, where each individual histogram (denoted by linestyle and color) corresponds to a different type of heating function. This means, for example, that the solid blue histogram includes emission measure ratios for all values of $t_N$, but for only those cases where heating events are chosen from a power-law distribution of $\alpha=-1.5$ and $t_{N,i}=t_N$ for all $i$ (i.e. $\beta=0$). We see that the emission measure ratio is largely insensitive to $\alpha$ and is peaked sharply at approximately $0.25$; the distribution peaks at slightly higher values for the $\beta=1$ case though the overall shape of the distribution is the same. Here we have not included the uniform heating case due to small number statistics (i.e. $<20$ emssion measure ratios). 
	%
	\par The right panel of \autoref{fig:em_ratio_hist_el} shows these same emission measure ratios, but now separated by $t_N$. For example, the solid red histogram includes emission measure ratios for every type of heating function (i.e. uniform, all $\alpha$ and all $\beta$), but for only those runs where $t_N=3000$ s. Note that we choose to only show results for five values of $t_N$ for aesthetic purposes. We find that for $t_N\le3000$ s, the emission measure ratios are largely clustered below 0.5, with $t_N=2000,3000,4000$ s peaking between $\sim0.3-0.45$. For $t_N>4000$ s, the emission measure ratio seems to be more sensitive to the heating frequency .
	%
	\par Similarly, we compute the same emission measure ratio for the case in which only the ions are heated. The left panel of \autoref{fig:em_ratio_hist_ion} shows histograms for each type of heating function. Similar to the electron heating results, we find that the $\beta=0$ distributions peak at lower values compared to the $\beta=1$ distributions and that in the $\beta=0$ case, the distribution of emission measure ratios is relatively insensitive to $\alpha$. However, unlike the electron heating case, the distributions in all cases are much wider and peak at higher values, $\sim0.5$ for $\beta=0$ and $>1$ for $\beta=1$. Furthermore, in the $\beta=1$ case, the $\alpha=-2.5$ distribution peaks at lower values compared $\alpha=-1.5,-2.0$. The right panel of \autoref{fig:em_ratio_hist_ion} shows the same emission measure ratios now grouped by waiting time where again we only show five values of $t_N$. With the exception of the $t_N=2000$ s and $t_N=3000$ s group, the distributions for each value of $t_N$ are much more spread out (as compared \autoref{fig:em_ratio_hist_el}). In particular, $t_N=5000$ s peaks above 1.0 and extends past 1.5 while $t_N=1000$ peaks near 0 and drops off very steeply before 0.5. In general, the emission measure ratios in the ion heating case appear to be more sensitive to $t_N$ compared to the electron heating case.
	%
	\par\authorcomment1{More interpretation here? Discuss conceptually the implication(s) of some of these results.}
	%%
	\section{Discussion}
	\label{sec:discussion}
	%%
	\begin{enumerate}
	\item \authorcomment2{$>10$ MK present in nanoflare train runs relative to single pulse results from paper I}
	\item \authorcomment2{consistency of results with Cargill (2014) on the cool side; unsuitability of the hot emission measure slope}
	\item \authorcomment2{importance of density preconditioning in departures from IEQ} 
	\item \authorcomment2{emission measure ratio results in light of Brosius et al, 2014}
	\item \authorcomment2{can we say something about possible heating frequencies? likelihood of electron heating over ion heating?}
	\end{enumerate}
	%
	\par\authorcomment1{Adding this paragraph here for now. Maybe chop up and incorporate into discussion or delete altogether} Another important difference between the uniform and power-law cases is the impact of NEI on the hot part of $\mathrm{EM}(T)$. Recall from \autoref{subsec:params} that the uniform heating rate $H_0=2t_{total}H_{eq}/N\tau=2H_{eq}(t_N/\tau+1)$. From the top panel of \autoref{fig:nT_sample_profiles} we see that in the power-law case, the heating rate can be as high as $H_i\sim0.3$ erg cm$^{-3}$ s$^{-1}$. In terms of a uniform heating rate, this corresponds to a waiting time of $t_N\approx8100$ s. So even for the lowest heating frequency in the uniform case, the heating rate is still much lower. Furthermore, for $t_N=5000$ s, the lowest heating frequency considered here, the loop is allowed to drain to its equilibrium value (as determined by the background heating) between each event such that events with a heating rate $H_0\approx0.2$ erg cm$^{-3}$ s$^{-1}$ are injected into a low-density loop, meaning that the ionization state will not be able to keep pace with the subsequently rapid rise in temperature, resulting in $\mathrm{EM}(T_{eff})$ having a lower temperature cutoff. On the other hand, when the heating events are selected from a power-law distribution, events with $H_i>0.2$ erg cm$^{-3}$ s$^{-1}$ can be injected into a much higher density loop (because lower values of $t_N$ prevent longer draining times). This allows the ionization state to be relatively representative of the electron temperature such that $\mathrm{EM}(T)\approx\mathrm{EM}(T_{eff})$.
	%
	\section{Conclusions}
	\label{sec:conclusions}
	%
	%
	\acknowledgments
	%
	WTB was provided travel support to the Coronal Loops Workshop VII held in Cambridge, UK, July 21-23, 2015, at which a preliminary version of this work was presented, by NSF award number 1536094. This work was supported in part by the Big-Data Private-Cloud Research Cyberinfrastructure MRI-award funded by NSF under grant CNS-1338099 and by Rice University.
	%
	\software{IPython \citep{perez_ipython:_2007}, Jupyter notebook, matplotlib \citep{hunter_matplotlib:_2007}, NumPy, seaborn \citep{waskom_seaborn:_2016}, Scipy \citep{van_der_walt_numpy_2011}}
	%
	%
	%Bibliography
	\bibliography{astrophys-abbrev.bib,references.bib}
	\bibliographystyle{aasjournal}
	%\bibliography{hot_non-flaring_plasma_2.bbl}
\end{document}